%%%%%%%%%%%%%%%%%%%%%%%%%%%%%%%%%%%%%%%%%
% "ModernCV" CV and Cover Letter
% LaTeX Template
% Version 1.1 (9/12/12)
%
% This template has been downloaded from:
% http://www.LaTeXTemplates.com
%
% Original author:
% Xavier Danaux (xdanaux@gmail.com)
%
% License:
% CC BY-NC-SA 3.0 (http://creativecommons.org/licenses/by-nc-sa/3.0/)
%
% Important note:
% This template requires the moderncv.cls and .sty files to be in the same 
% directory as this .tex file. These files provide the resume style and themes 
% used for structuring the document.
%
%%%%%%%%%%%%%%%%%%%%%%%%%%%%%%%%%%%%%%%%%

%----------------------------------------------------------------------------------------
%	PACKAGES AND OTHER DOCUMENT CONFIGURATIONS
%----------------------------------------------------------------------------------------

\documentclass[10pt,a4paper,sans]{moderncv} % Font sizes: 10, 11, or 12; paper sizes: a4paper, letterpaper, a5paper, legalpaper, executivepaper or landscape; font families: sans or roman
\usepackage{standalone}
\moderncvstyle{classic} % CV theme - options include: 'casual' (default), 'classic', 'oldstyle' and 'banking'
\moderncvcolor{blue} % CV color - options include: 'blue' (default), 'orange', 'green', 'red', 'purple', 'grey' and 'black'

\usepackage{lipsum} % Used for inserting dummy 'Lorem ipsum' text into the template

\usepackage[scale=0.85]{geometry} % Reduce document margins
%\setlength{\hintscolumnwidth}{3cm} % Uncomment to change the width of the dates column
%\setlength{\makecvtitlenamewidth}{10cm} % For the 'classic' style, uncomment to adjust the width of the space allocated to your name

%\usepackage[utf8]{inputenc}

%\usepackage{booktabs}
\usepackage{fontawesome}
\usepackage{marvosym} % For cool symbols.
%\usepackage{hyperref}



%----------------------------------------------------------------------------------------
%	NAME AND CONTACT INFORMATION SECTION
%----------------------------------------------------------------------------------------

\firstname{Mahdi(Sum)} % Your first name
\familyname{Qezlou} % Your last name

% All information in this block is optional, comment out any lines you don't need
\title{Curriculum Vitae}
\address{Department of Physics and Astronomy}{University of California, Riverside}


%\social{github}{stefano-bragaglia}
\email{mahdi.qezlou@email.ucr.edu} 



\homepage{qezlou.github.io/}{Webpage}

% social link \faGithub, \faSkype, \faLinkedin,\faStackExchange, and \faStackOverflow
\extrainfo{
    \faGithub\href{https://github.com/qezlou}{ Github} \quad
    %\faLinkedin\href{https://www.linkedin.com/abc/}{ Linkedin} \quad
    %\faSkype\href{https://skype.com/abc}{Skype}
    }



%\social[linkedin][www.linkedin.com]{name}
% The first argument is %the url for the clickable link, the second argument is the url displayed in the %template - this allows special characters to be displayed such as the tilde in this %example

\photo[70pt][0.3pt]{photo} % The first bracket is the picture height, the second is %the thickness of the frame around the picture (0pt for no frame)
%\quote{Not Attention, Patience is all we need.}

%----------------------------------------------------------------------------------------

\newcommand{\cvdoublecolumn}[2]{%
  \cvitem[.75em]{}{%
    \begin{minipage}[t]{\listdoubleitemcolumnwidth}#1\end{minipage}%
    \hfill%
    \begin{minipage}[t]{\listdoubleitemcolumnwidth}#2\end{minipage}%
    }%
}



\usepackage{multibbl}
\newcommand\Colorhref[3][orange]{\href{#2}{\small\color{#1}#3}}


% \newcommand{\cvreference}[7]{%
%     \textbf{#1}\newline% Name
%     \ifthenelse{\equal{#2}{}}{}{\addresssymbol~#2\newline}%
%     \ifthenelse{\equal{#3}{}}{}{#3\newline}%
%     \ifthenelse{\equal{#4}{}}{}{#4\newline}%
%     \ifthenelse{\equal{#5}{}}{}{#5\newline}%
%     \ifthenelse{\equal{#6}{}}{}{\emailsymbol~\texttt{#6}\newline}%
%     \ifthenelse{\equal{#7}{}}{}{\phonesymbol~#7}}

\begin{document}

\makecvtitle % Print the CV title




%----------------------------------------------------------------------------------------
%	EDUCATION SECTION
%----------------------------------------------------------------------------------------

\section{Employment history and Education}

\cventry{2024-present}{Postdoctoral Fellow}{University of Texas, Austin}{}{}
{Inference on cosmological parameters through modeling the observed distribution of galaxies with the \Colorhref{https://hetdex.org/}{HETDEX} observations}

\cventry{2018--2024}{PhD, Physcis \& Astronomy}{University of  California, Riverside}{}{}
{Computational Astrophysics: Lyman-$\alpha$ forest tomography, Line Intensity Map, Machine Learning } 

\cvitem{Advisors and mentors:}{Simeon Bird, UCR.
Andrew Newman, Carnegie Observatories. 
Gwen Rudie, Carnegie Observatories,
Adam Lidz, UPenn}

\cventry{2013-2018 :}{B.Sc in Physics }{Sharif University of Technology, SUT}{}{}{}
\cvitem{Research Advisor :}{Shant Baghram, SUT}

%{Advanced exposure to various areas of computer science along with a one and half year research project on Reversible Logic Synthesis.}
%\cvitem{CGPA :}{7.96/10}
%\cventry{2009--2013 :}{Bachelor of Engineering, Computer Science \& Technology}{Indian Institute of Engineering Science \& Technology}{Shibpur(\textit{Formerly} Bengal Engineering and Science University, Shibpur)}{}{}
%{Comprehensive exposure to the core areas of Computer Science along with a final year project on Data-mining}
%\cvitem{CGPA :}{7.36/10}
% \cventry{2008 :}{Higher Secondary Examination}{Belmuri Union Institution}{Belmuri}{}{ Mathematics, Physics, Chemistry, Biology, English, Bengali}
% {}
% \cvitem{Percentage :}{81.2 \%}
% \cventry{2006 :}{Secondary Examination}{Belmuri Union Institution}{Belmuri}{}{ Mathematics, Physical Science, Life Science, Geography, History, English, Bengali}
% {}
% \cvitem{Percentage :}{90.8 \%}


\section{Research Interests:}
 
\cventry{}{ Cosmological hydrodynamic simulations (\texttt{MP\_GADGET}),  $Ly-\alpha$ forest tomography at cosmic noon, Modeling Line Intensity Map Signal, Machine Learning, Artificial Intelligence, Bayesian statistics}{}{}{}{}

%----------------------------------------------------------------------------------------
%	PUBLICATION SECTION
%----------------------------------------------------------------------------------------


\section{Publications}
\cvitem{on ADS}{\Colorhref{https://ui.adsabs.harvard.edu/public-libraries/UjaV3zMmSmGOO7h_FdRJlA}{https://ui.adsabs.harvard.edu/public-libraries/UjaV3zMmSmGOO7h_FdRJlA}} 

\subsection{Selected Published}
\newbibliography{selected}
\bibliographystyle{selected}{plainyrrev}
\nocite{selected}{*}
\bibliography{selected}{selected}
{\large \textsc{Refereed Journal Articles}}

\subsection{Other Published}
\newbibliography{journal}
\bibliographystyle{journal}{plainyrrev}
\nocite{journal}{*}
\bibliography{journal}{journal}
{\large \textsc{Refereed Journal Articles}}

\subsection{Submitted Articles}

\newbibliography{papers_submitted}
\bibliographystyle{papers_submitted}{plainyrrev}
\nocite{papers_submitted}{*}
\bibliography{papers_submitted}{papers_submitted}
{\large \textsc{Refereed Journal Articles}}








%----------------------------------------------------------------------------------------
%	WORK EXPERIENCE SECTION
%----------------------------------------------------------------------------------------


%----------------------------------------------------------------------------------------
%	Fellowships \& Awards
%----------------------------------------------------------------------------------------

\section{Fellowships \& Awards}
\cvitem{2023 -- 2024}{\textit{\textbf{Dissertation-Year Fellowship}}, Awarded to only 3 students at UCR among all PhD majors.}
\cvitem{2020 -- 2022}{\textit{\textbf{Carnegie-UCR Fellowship}} Graduate researcher fellow at Carnegie observatories to work on Ly$\alpha$ tomography IMACS survey (LATIS) project.}

\cvitem{2018 -- 2019}{\textit{\textbf{UCR Graduate Dean Fellowship}}, for Fall, spring and Summer quarters}

%----------------------------------------------------------------------------------------
%	COMPUTER SKILLS SECTION
%----------------------------------------------------------------------------------------

\section{Computing skills}

\cvitem{Computational}{Machine learning, Bayesian Statistics}
\cvitem{Programming}{Python, C, MPI parallel computing, High-performance computing}
\cvitem{Visualization}{Virtual Reality engines, e.g.  \textit{Blender} and \textit{Unity}, \Colorhref{https://www.youtube.com/@Qezlou}{YouTube Channel}}


%----------------------------------------------------------------------------------------
%	Position of Responsibility SECTION
%----------------------------------------------------------------------------------------

\section{Mentorship Experience}
\cventry{Fall-Winter 2022-23}{High-school science fair project, student:  Joseph Zenarosa (Martin Luther King High, Riverside)}{}{Reionization in \texttt{ASTRID} , a cosmological hydrodynamic simulation}{}{Mentoring the student for science fair competition}
\cventry{summer 2022}{Undergraduate summer project, student:  Kevin Hong (UCLA)}{}{\textit{3D Visualization of cosmological hydrodynamical simulations}}{}{Mentoring student, visualizations using \texttt{Blender} open-source software}
\cventry{summer 2021 and 2022 and 2023}{CASSI, Summer research program for undergraduates at Carnegie observatory,  }{}{\textit{Teaching python, high-performance computing, and scientific visualizations to $\sim$ 20 students each year}}{}{}


%----------------------------------------------------------------------------------------
%	Talks
%----------------------------------------------------------------------------------------

\section{Talks}

\cvitem{Fall}
{Cosmology and galaxy evolution with Ly-alpha tomography and Line Intensity Map} 
\cvitem{2023}{Harvard, MIT, University of Pennsylvania, University of California Irvine, University of California Santa Barbara, University of Texas Austin}
\vspace{3mm}



\cvitem{February}
{Presenting Tutorial on Machine Learning approaches in large-scale galaxy formation simulations} 
\cvitem{2023}{\Colorhref{https://datadrivengalaxyevolution.github.io/}{KITP Program, Data Driven Astronomy}}
\vspace{3mm}

\cvitem{December}{Boosting Line Intensity Map Signal-to-Noise with the Ly-$\alpha$ Forest Cross-Correlation }
\cvitem{2023}{Flatiron Institute, Cosmology and Astrophysics with Machine Leaning Simulations(CAMELS) \Colorhref{https://indico.flatironinstitute.org/event/3324/timetable/?view=standard#15-lyman-alpha-tomography-at-c}{workshop}}
\vspace{3mm}

\cvitem{October}{Characterizing Protoclusters and Protogroups at z $\sim$ 2.5 Using Ly-$\alpha$ Tomography}
\cvitem{2022}{\Colorhref{https://www.ipac.caltech.edu/event/575}{IPAC Talk Series}}
\vspace{3mm}
\cvitem{Jun}{Characterizing Protoclusters and Protogroups at z $\sim$ 2.5 Using Ly-$\alpha$ Tomography}
\cvitem{2022}{\Colorhref{https://www.cosmologyfromhome.com/}{Cosmology from home conference}}

\vspace{3mm}
\cvitem{September}{Characterizing Protoclusters and Protogroups at z $\sim$ 2.5 Using Ly-$\alpha$ Tomography}
\cvitem{2022}{\Colorhref{https://protoclusters.wordpress.com/home/program/}{Protoclusters: galaxies in confinement}}


%----------------------------------------------------------------------------------------
%	Professional service
%----------------------------------------------------------------------------------------

\section{Professional service }

\cvitem{}{Referee for high-impact journals: \textbf{ ApJ Letters} \textbf{ Physical Review D}}
\cvitem{}{Review panelist : \textbf{Gemini telescope} Canadian time allocation committee (CanTAC)}



%----------------------------------------------------------------------------------------
%	Teaching Assistantship SECTION
%----------------------------------------------------------------------------------------

\section{Teaching Assistantship}
\cventry{2018 :}{Physics lab I}{}{UCR}{}{}
\cventry{2017-18 :}{Quantum mechanics I \& II}{}{SUT}{}{}
\cventry{2016 :}{Special relativity}{}{}{SUT}{}

\end{document}